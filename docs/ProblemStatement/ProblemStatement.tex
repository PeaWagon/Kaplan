\documentclass{article}

\usepackage{tabularx}
\usepackage{booktabs}

\title{{CAS 741: Problem Statement}\\{\Large Determination of Conformational 
Isomers}}

\author{\textbf{Jen Garner} (PeaWagon//garnej2)}

\date{Due: September 14, 2018 (Friday)}

\pagenumbering{gobble}

\input{../Comments}

\begin{document}

\maketitle

\begin{table}[hp]
\caption{Revision History} \label{TblRevisionHistory}
\begin{tabularx}{\textwidth}{llX}
\toprule
\textbf{Date} & \textbf{Developer(s)} & \textbf{Change}\\
\midrule
September 13, 2018 (Thursday) & J. Garner & Creation of document. \\
September 14, 2018 (Friday) & J. Garner & Make changes as per issues in Github. 
\\
\bottomrule
\end{tabularx}
\end{table}

\noindent{\large\textbf{Motivation for Project}}

\vspace{2mm} %2mm vertical space

Computational chemists use software to predict the properties of molecules pertinent to their research or profession. Before running calculations, a starting geometry is required. This initial configuration can be random, or it can be based on, for example, a crystal structure. Reasonable input geometries are important, since some computational methods will not converge (or will take much longer to converge) given a poor input structure. 

A number of reasonable input structures may exist, especially for molecules with many rotatable bonds. These structures are known as conformational isomers, and they represent the local minima of the potential energy surface. Since conformational isomers do not require the breaking or formation of bonds, their lifetimes are very short; therefore, they are suitable for computational study.

\vspace{3mm} %3mm vertical space

\noindent{\large\textbf{Proposal}}

\vspace{2mm} %2mm vertical space

This program will construct a set of conformers for each molecule in a list of 
input molecules. The program will convert a SMILES (simplified molecular-input 
line-entry system) string into an initial population of possible geometries, as 
represented by lists of dihedral angles. Then, an optimisation, such as an 
evolutionary algorithm, will be implemented to search the energy surface and 
locate conformers.

\vspace{3mm} %3mm vertical space
\newpage
\noindent{\large\textbf{Environment \& Usage}}

\vspace{2mm} %2mm vertical space

The author will try to package this program using Anaconda, so that it might be 
installed in a Conda environment. Python is likely going to be the language of 
choice, such that no compiling step will be needed (for initial stages - future 
work may be to profile the code and replace slower areas with a compiled 
language such as C++ or Rust).

Attempts will be made to locate an open-source software package for evaluating 
the energy of a given structure (to confirm the location of an energy minimum). 
For future research, the program will be linked to Vetee - a work-in-progress 
project to which the author is a contributor. Vetee handles databases of 
molecules and is responsible for setting-up and running calculations using 
Gaussian. Since Gaussian is proprietary software, the energy calculation step 
should be left open such that the user is not restricted behind a paywall.

This program will be used by the author to conduct research into databases of 
floppy organic molecules, polycyclic aromatic hydrocarbons, carbenes, and 
common drug molecules. The program will have to run on high performance 
computing (HPC) clusters (Sharcnet, Compute Canada, etc.), which almost 
exclusively use Linux OS. A serial version will be written first to ensure that 
the project is executable on personal computers. Gaussian is available on the 
HPC cluster, Graham, and its use is dependent on having a Compute Canada 
account and in agreeing to the following terms:
\begin{enumerate}
	\item {I am not a member of a research group developing software 
		competitive to Gaussian.}
	\item {I will not copy the Gaussian software, nor make it available to 
		anyone else.}
	\item {I will properly acknowledge Gaussian Inc. and Compute Canada in 
		publications.}
	\item {I will notify Compute Canada of any change in the above 
		acknowledgement.}
\end{enumerate}

Ultimately, this program should be usable by other chemists to determine good 
initial structures for molecules where the structure is unknown. The benefit of 
using this program over other geometry optimisation techniques will be the 
production of a set of possible geometries, rather than a single solution. For 
example, some drug molecules may adopt certain conformations in active sites 
that may not be the global minimum of the energy surface. The program should be 
at least as fast as performing a geometry optimisation with a high level of 
quantum mechanical theory (coupled-cluster, full configuration interaction, 
etc.).



%Put your problem statement here.  Comments to you can be added, like this:

%\wss{comment}

%You can also leave comments for yourself, like this:

%\an{comment}

\end{document}
