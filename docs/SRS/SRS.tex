\documentclass[12pt]{article}

\usepackage{amsmath, mathtools}
\usepackage{amsfonts}
\usepackage{amssymb}
\usepackage{graphicx}
\usepackage{colortbl}
\usepackage{xr}
\usepackage{hyperref}
\usepackage{longtable}
\usepackage{xfrac}
\usepackage{tabularx}
\usepackage{float}
\usepackage{siunitx}
\usepackage{booktabs}
\usepackage{caption}
\usepackage{pdflscape}
\usepackage{afterpage}

\usepackage[round]{natbib}

% jen things
\usepackage[version=3]{mhchem} % electron symbol
%	\ce{n^0} \ce{e-} \ce{p+}
\usepackage{braket} % bra-ket notation
\usepackage{textcomp} % degree symbol \textdegree

%\usepackage{refcheck}

\hypersetup{
    bookmarks=true,         % show bookmarks bar?
      colorlinks=true,       % false: boxed links; true: colored links
    linkcolor=red,          % color of internal links (change box color with linkbordercolor)
    citecolor=green,        % color of links to bibliography
    filecolor=magenta,      % color of file links
    urlcolor=cyan           % color of external links
}

\input{../Comments}

% For easy change of table widths
\newcommand{\colZwidth}{1.0\textwidth}
\newcommand{\colAwidth}{0.13\textwidth}
\newcommand{\colBwidth}{0.82\textwidth}
\newcommand{\colCwidth}{0.1\textwidth}
\newcommand{\colDwidth}{0.05\textwidth}
\newcommand{\colEwidth}{0.8\textwidth}
\newcommand{\colFwidth}{0.17\textwidth}
\newcommand{\colGwidth}{0.5\textwidth}
\newcommand{\colHwidth}{0.28\textwidth}

% left is what you call with latex, right is output

% Used so that cross-references have a meaningful prefix
\newcounter{defnum} %Definition Number
\newcommand{\dthedefnum}{GD\thedefnum}
\newcommand{\dref}[1]{GD\ref{#1}}
\newcounter{datadefnum} %Datadefinition Number
\newcommand{\ddthedatadefnum}{DD\thedatadefnum}
\newcommand{\ddref}[1]{DD\ref{#1}}
\newcounter{theorynum} %Theory Number
\newcommand{\tthetheorynum}{T\thetheorynum}
\newcommand{\tref}[1]{T\ref{#1}}
\newcounter{tablenum} %Table Number
\newcommand{\tbthetablenum}{T\thetablenum}
\newcommand{\tbref}[1]{TB\ref{#1}}
\newcounter{assumpnum} %Assumption Number
\newcommand{\atheassumpnum}{P\theassumpnum}
\newcommand{\aref}[1]{A\ref{#1}}
\newcounter{goalnum} %Goal Number
\newcommand{\gthegoalnum}{P\thegoalnum}
\newcommand{\gsref}[1]{GS\ref{#1}}
\newcounter{instnum} %Instance Number
\newcommand{\itheinstnum}{IM\theinstnum}
\newcommand{\iref}[1]{IM\ref{#1}}
\newcounter{reqnum} %Requirement Number
\newcommand{\rthereqnum}{P\thereqnum}
\newcommand{\rref}[1]{R\ref{#1}}
\newcounter{lcnum} %Likely change number
\newcommand{\lthelcnum}{LC\thelcnum}
\newcommand{\lcref}[1]{LC\ref{#1}}

\newcommand{\progname}{Kaplan} % PUT YOUR PROGRAM NAME HERE

\usepackage{fullpage}

\begin{document}

\title{Kaplan Software Requirements Specification for Conformer Searching} 
\author{Jen Garner}
\date{\today}
	
\maketitle

~\newpage

\pagenumbering{roman}

\section{Revision History}\label{rev-hist}

\begin{tabularx}{\textwidth}{p{3cm}p{2cm}X}
\toprule {\bf Date} & {\bf Version} & {\bf Notes}\\
\midrule
October 4, 2018 (Thursday) & 1.0 & first draft for 
submission \\
October 22, 2018 (Monday) & 1.1 & enumerate 
non-functional requirements, add dictionary to 
Texstudio and fix spelling errors \\
October 27, 2018 (Saturday) & 1.2 & fix some typos and consistency errors from 
github issues\\
November 14, 2018 (Wednesday) & 1.3 & remove original wss comments, move 
numerical constants table to appendix, fix traceability issues \\
November 21, 2018 (Wednesday) & 1.4 & fix some issues from Github \\
\bottomrule
\end{tabularx}

~\newpage

\section{Reference Material}\label{section-ref-tables}

Units, symbols, abbreviations, and acronyms are abbreviated here, as 
used in this document. Numerical constants are found in the appendix, Section 
\ref{num-const}.

\subsection{Table of Units and Constants}

This section describes the units that are used in the document. For each unit, 
the symbol is given followed by a description of the unit and its name.
~\newline

\renewcommand{\arraystretch}{1.2}
%\begin{table}[ht]
  \noindent \begin{tabular}{l l l} 
    \toprule		
    \textbf{Symbol} & \textbf{Unit} & \textbf{Name}\\
    \midrule 
    \si{\metre} & length & metre\\
    \si{\angstrom} & length & $1\si{\angstrom} = 1$x$10^{-10}$\si{\metre} \\
    \si{\kilogram} & mass	& kilogram\\
    \si{\second} & time & second\\
    \si{\joule} & energy & Joule\\
    \si{\farad} & electrical capacitance & farad \\
    \si{\coulomb} & electric charge & coulomb (\si{\ampere\second}) \\
    \bottomrule
  \end{tabular}
  %	\caption{Provide a caption}
%\end{table}



\subsection{Table of Symbols}\label{SRS-symbols}

The table that follows summarizes the symbols used in this document along with
their units.

\renewcommand{\arraystretch}{1.2}
%\noindent \begin{tabularx}{1.0\textwidth}{l l X}
\noindent \begin{longtable*}{l l p{12cm}} \toprule
\textbf{symbol} & \textbf{unit} & \textbf{description}\\
\midrule 

$n_a$ & unitless & number of atoms in the input molecule \\
$n_G$ & unitless & number of conformers (distinct geometries) being 
simultaneously optimised \\
$G_i$ & \si{\angstrom} & geometry for conformer $i$; 
matrix of Cartesian 
coordinates with shape($n_a$, 3) \\
$D_i$ & \textdegree & list of dihedral angles for conformer $i$; array of 
length $n_a - 3$ \\
$S_E$ & \si{\joule} & sum of conformer potential energies \\
$C_E$ & $\frac{1}{\si{\joule}}$ & coefficient for $S_E$ in the fitness function 
\\
$S_\text{RMSD}$ & \si{\angstrom} & sum of root-mean-square distances for all 
conformer geometries \\
$C_\text{RMSD}$ & $\frac{1}{\si{\angstrom}}$ & coefficient for $S_\text{RMSD}$ in the 
fitness function \\
$Fit_G$ & unitless & the fitness of the set of conformers \\
$E$ & \si{\joule} & energy (of conformer) \\
$\hat{H}$ & \si{\joule} & Hamiltonian operator \\
$\Psi$ & unitless & wavefunction \\
$\sigma$ & unitless & spin ($\alpha$ or $\beta$) \\
$\phi$ & unitless & atomic orbital \\
$r$ & \si{\angstrom} & position of electron in Cartesian coordinates \\
$R$ & \si{\angstrom} & position of nuclei in Cartesian coordinates \\
$c_i$ & tbd & basis set constant \\
$c_N$ & tbd & normalization/principle quantum number constant \\
$\zeta$ & tbd & effective nuclear charge constant \\
\ce{e-} & N/A & electron \\
x, y, z & \si{\angstrom} & components of atomic Cartesian coordinates \\
X, Y, Z & \si{\angstrom} & components of electronic Cartesian coordinates \\
$Z_a$ & unitless & atomic number \\
i, j, k & N/A & indexing variables (see text for specific use case) \\
\end{longtable*}

\subsection{Abbreviations and Acronyms}

\renewcommand{\arraystretch}{1.2}
\begin{tabular}{l l} 
  \toprule
  \multicolumn{2}{c}{\large{Document-Specific Acronyms}} \\
  \midrule
  \textbf{Symbol} & \textbf{Description}\\
  \midrule 
  A & Assumption\\
  DD & Data Definition\\
  GD & General Definition\\
  GS & Goal Statement\\
  IM & Instance Model\\
  LC & Likely Change\\
  PS & Physical System Description\\
  R & Requirement\\
  SRS & Software Requirements Specification\\
  \progname{} & Program Name \\
  T & Theoretical Model\\
  \midrule
  \multicolumn{2}{c}{\large{Chemical Acronyms}} \\
  \midrule
  \textbf{Symbol} & \textbf{Description}\\
  \midrule
  BO & bond order \\
  SMILES & simplified molecular-input line-entry system \\
  AO & atomic orbital \\
  MO & molecular orbital \\
  LCAO & linear combination of atomic orbitals \\
  STO & slater-type orbital \\
  GTO & Gaussian-type orbital \\
  QCM & quantum chemical method \\
  BS & basis set \\
  RMSD & root-mean square deviation \\
  VSEPR & valence shell electron pair repulsion \\
  VB & valence bond \\
  3D & three-dimensional \\
  \bottomrule
\end{tabular}\\

\newpage

\tableofcontents

~\newpage

\pagenumbering{arabic}

\section{Introduction} \label{intro}

Molecular geometry is a necessary piece of information with regards to running 
calculations. Most computational chemists will run a 
geometry optimization for 
their molecule of interest before they can investigate 
other properties, such as 
charge distribution and interaction energies. Without breaking any bonds, there 
are multiple ways in which the atoms can be positioned such that an optimal 
geometry is obtained - these geometries are called conformational isomers 
(sometimes abbreviated as conformers). Rather than 
performing an exhaustive 
search of all possible geometries, \progname{} is a 
package designed for the 
efficient exploration and optimization of molecular 
geometries, with the end goal 
of procuring a set of conformers.
A classic example of conformers is the boat versus the chair 
conformation of cyclohexane (see Figure \ref{C6H14}).

\begin{figure}[H]
	\begin{center}
	\includegraphics[]{C6H14}
	\end{center}
	\caption{Two examples of conformational isomers of cyclohexane.}
	\label{C6H14}
\end{figure}


Conformers are important for determining rates of interconversion and reaction, 
stereochemistry of products and intermediates, and mechanisms. Knowing the
likelihood for a molecule to adapt one conformation over another would indicate
a preference for producing a particular product. Other interesting questions
include those relating to guest-host chemistry and macromolecules (enzymes and substrates)
 - what happens with increasing
pressure or temperature, or when using different solvents? Does the structure
change to adapt a new conformation? If there are multiple semi-stable structures,
how many of these minima exist, and which ones do I care about for my given reaction?


 A simple geometry 
optimization of cyclohexane would only afford a single conformer, but, in using \progname{}, 
it will be possible to 
find multiple local minima of this molecule. The conformer search proceeds mainly through the
manipulation of dihedral angles.

\subsection{Purpose of Document}

This document outlines the requirements that the software, \progname{}, must 
meet. It is an abstract document that does not describe the details of 
implementation. Rather, a discussion of its intended functionalities, 
performance, goals, and qualities will be presented. This document will be used 
as a reference guide when writing the software design specification and the 
software verification and validation plan. The inputs, expected outputs, and 
user characteristics for the program will be outlined, as well as the theory 
needed for searching a molecular geometry space. Any assumptions inherent to 
solving the problem will be mentioned in this document. 

\subsection{Scope of Requirements}
Kaplan will optimize the geometry of an input molecule 
to find its conformers 
by searching the potential energy surface. The size of molecule that can be 
accommodated depends greatly on the other inputs of the system, such as level 
of theory and convergence criteria, but should be held within reasonable bounds 
such as a maximum of a couple hundred atoms. During 
optimization, the bond 
lengths and bond angles will be held fixed, and only 
the dihedral angles will 
be manipulated. As a result, molecules with less than 4 atoms will remain 
unchanged after optimization (no dihedral angles). 
Furthermore, the conformer 
with the lowest energy is not likely to represent the true minimum \wss{global
  minimum?} of the 
potential energy landscape without subsequent 
optimization of the other bond 
angles/lengths. Lastly, the system is not intended to be impervious to abnormal 
or forced bonding behaviour.

\subsection{Characteristics of Intended Reader} 
The reader should have taken first-year undergraduate chemistry, physics, and 
mathematics. They should be familiar with some bonding theory in molecules 
(valence bond (VB) theory, molecular orbital (MO) theory, and valence-shell 
electron pair repulsion (VSEPR)) and comfortable with the concept of 
optimization (especially of a multi-dimensional 
surface). 

To understand the energy calculations, the reader should have taken a quantum 
physics or quantum chemistry undergraduate course. Having a basic understanding 
the following terms will be useful:
\begin{itemize}
	\item Schr\"{o}dinger equation
	\item Hartree Fock
	\item wavefunction
	\item basis set
	\item Hamiltonian operator
	\item restricted versus unrestricted quantum chemistry calculations
\end{itemize}

\subsection{Organization of Document}

This document follows the template outlined in \citet{SmithAndLai2005, 
SmithEtAl2007}. Section 
\ref{rev-hist} is the revision history, including some updates as to how the 
document has changed over time. Section \ref{section-ref-tables} 
provides tables for all of the units, symbols, acronyms, and 
abbreviations used throughout the document. Section \ref{intro} is an overview 
of the purpose and scope of the system, including an 
explanation the intended 
reader for this document. Section \ref{sys-desc} goes into more detail 
regarding the system and its inputs, and describes the responsibilities of the 
user versus the program. Some constraints are also mentioned here for the 
design, and the user characteristics for the program are given. 

Section \ref{section-system-desc} defines the problem to be solved, the 
terminology and definitions for the problem, the physical system that will be 
used to represent the problem, the goal statements, the characteristics of the 
solution, the models that are used to solve the problem, the assumptions about 
the problem, and the constraints on the output.

Section \ref{section-require} lists the functional and non-functional 
requirements of the program. Section \ref{section-lc} gives the likely changes 
to the program, and Section \ref{section-trace} gives traceability tables 
showing how the portions of the document are connected. If changes are made to 
the document, then these traceability tables will be referenced to determine 
what other changes might be necessary in order to satisfy requirements.

\section{General System Description} \label{sys-desc}

The interactions between the system and its environment are discussed in this 
section. The user characteristics and system constraints are also given.

\subsection{System Context}

\noindent Here the system context for Kaplan is shown (Figure 
\ref{sys-context}). With 
respect to inputs, convergence conditions include number of expected 
conformers, energetic requirements, and the method and basis set used to 
perform energy calculations. As for outputs, an energy 
will be returned for each 
conformer geometry. 

%It should be noted here that the program will not 
%calculate the energy; rather, an external library (examples: Gaussian 
%\ref{g16}, Psi4 \ref{psi4}, Horton \ref{horton}) will be called upon to do the 
%calculations. The user will not have to interact with this external library, 
%but they may have to provide input formatted to that library's specification 
%(such instructions will be included as part of Kaplan's documentation). 


\begin{figure}[H]
	\includegraphics[width=\textwidth]{sys-context}
	\caption{The circles represent user interaction, and the rectangle 
	represents the program. The inputs are fed to the program, and the 
	outputs are given back to the user, as indicated by the arrows.}
	\label{sys-context}
\end{figure}

\begin{itemize}
\item User Responsibilities:
\begin{itemize}
\item provide chemically and computationally reasonable input

%provide an input molecule that is chemically reasonable (i.e. likely to 
%converge during optimization)
%give a level of theory (quantum mechanical method and basis set) for 
%evaluating the energy
%\item make sure the method and basis set make chemical and computational sense 
%for the input molecule
\item set the convergence criteria (how long will the program search for 
conformers? What energetic requirements should the 
conformers possess? Estimate 
the number of expected conformers, etc.)
\end{itemize}
\item \progname{} Responsibilities:
\begin{itemize}
\item prepare molecular geometry for optimization
%create initial coordinates for the molecule and generate a list of 
%dihedral angles
%convert between various input geometry types (examples: SMILES string, 
%z-matrix, xyz (Cartesian) coordinates)
\item find conformers and calculate their energies
\item determine if the convergence conditions have been met
%, such as finding enough local minima (finding sufficient conformers)
\item ensure that returned conformers have distinct geometries 
\end{itemize}
\end{itemize}

\subsection{User Characteristics} \label{SecUserCharacteristics}

The user of \progname{} should have taken first-year undergraduate physics, 
chemistry, and mathematics. The user must understand the 
impact of changing the quantum mechanical method and basis set for energy 
calculations, which implies that they understand basic quantum mechanics 
(including how to solve the Schr\"{o}dinger equation - third-year quantum 
physics 
course). The user should have a sense of whether their input geometry (where 
applicable) will converge under optimization

%They must be familiar with how to write a script such that they can interface
%with the program. 

\subsection{System Constraints}

An evolutionary algorithm will be used to search the potential energy space. 
Energies will be calculated using an open-source quantum chemistry package.

\section{Specific System Description} \label{section-system-desc}

This section first presents the problem description, which gives a high-level
view of the problem to be solved.  This is followed by the solution 
characteristics
specification, which presents the assumptions, theories, definitions and finally
the instance models.  

\subsection{Problem Description} \label{Sec_pd}

\progname{} is designed to search the potential energy surface for 
conformational isomers of an input molecule by manipulating dihedral angles. 
The solution to a fitness function, which is related to the energies of and 
spatial differences between conformers, will be found and maximized. As 
calculated using a quantum mechanical approach, the energies of the conformers 
will be optimized when they have the lowest energy (largest negative value). 
The set of conformers will be optimized when the RMSD (root-mean-squared 
deviation) is maximized since this is a measure of how different the conformers 
are from one another. If the user specifies that they wish to obtain 3 
conformers, the RMSD measure is necessary, otherwise the user could get 3 
global minima of the potential energy space representing the exact same 
molecule. 

\subsubsection{Terminology and Definitions}

This subsection provides a list of terms that are used in the subsequent
sections and their meaning, with the purpose of reducing ambiguity and making it
easier to correctly understand the requirements:

\begin{itemize}
\item \textbf{molecule:} a collection of atoms that are related in space, 
either covalently or non-covalently.
\item \textbf{bond angle:} the angle formed between three atoms.
\item \textbf{bond length:} the distance between the centre of masses of two 
atoms.
\item \textbf{dihedral angle} the angle between two intersecting planes, where 
each plane bisects 3 atoms. See Figure \ref{butane-dihedral} for an example.
\item \textbf{bond order:} the number of chemical bonds connecting two atoms. 
This value sometimes depends on the theory used.
\item \textbf{conformational isomer:} molecules with the same number of 
atoms that are related by free rotation*. May also be 
referenced as conformer in the text. 
\item \textbf{quantum chemical method:} the strategies used to solve the 
Schr\"{o}dinger equation.
\item \textbf{basis set:} how to describe the molecule mathematically such that 
the Schr\"{o}dinger equation can be evaluated.
\item \textbf{z-matrix:} an input file type that uses dihedral angles and 
connectivities rather than Cartesian coordinates.
\end{itemize}

*Here is a definition of free rotation, from \cite{stereochem-terms}. Free rotation 
can occur around single bonds, weak double bonds, triple bonds, and quintuple bonds.
\begin{quote}
In a stereochemical context the rotation about a bond is called `free' when the 
rotational barrier is so low that different conformations are not perceptible as 
different chemical species on the time scale of the experiment. The inhibition of 
rotation of groups about a bond due to the presence of a sufficiently large 
rotational barrier to make the phenomenon observable on the time scale of the 
experiment is termed hindered rotation or restricted rotation.
\end{quote}

\subsubsection{Physical System Description}

The physical system of \progname{} includes the following elements:

\begin{itemize}

\item[PS1:] Molecule for which to find conformers. This molecule is specified 
by a file, such as xyz (Cartesian coordinates) or a 
z-matrix, a SMILES string, 
or a name. From this geometry, the dihedral angles can be obtained.

\end{itemize}

\begin{figure}[H]
	\begin{center}
	\includegraphics[width=0.5\textwidth]{butane-dihedral}
	\end{center}
	\caption{The dihedral angle formed between the 4 highlighted atoms is 
	approximately 180\textdegree. Note that if the molecule was rotated such 
	that the blue-highlighted hydrogen atom on the left went behind the 
	blue-highlighted hydrogen atom on the right, then the dihedral angle would 
	then be 0\textdegree.}
	\label{butane-dihedral}
\end{figure}

An example of the physical system for the molecule butane ($C_4H_{10}$) can be 
found in Table \ref{input-butane}. A visual for the molecule itself is 
displayed in Figure \ref{butane-dihedral}.

\begin{table}[H]
	\begin{tabular}{llll}
		\toprule
		\textbf{Z-Matrix} & \textbf{XYZ (Cartesian)} & 
		\textbf{SMILES} & \textbf{Dihedral} \\
		& \textbf{Coordinates} & \textbf{String} & \textbf{Angles} \\
		\hline
	C   1 	&		14		&  CCCC & 178.8 \\
	C   1 1.527	&	butane  		&		& 238.9 \\
	C   1 1.520  2 111.489			& C -0.5630  0.5160  0.0071 &	& 121.2 \\
	C   2 1.520  1 111.497  3 178.8	& C  0.5630 -0.5159  0.0071	&	& 300.1 \\
	H   1 1.096  2 109.879  3 238.9	& C -1.9293 -0.1506 -0.0071	&	& 57.8 \\
	H   1 1.096  2 109.864  3 121.2	& C  1.9294  0.1505 -0.0071	&	& 299.7 \\
	H   2 1.096  1 109.866  3 300.1	& H -0.4724  1.1666 -0.8706	&	& 60.2 \\
	H   2 1.096  1 109.874  3  57.8	& H -0.4825  1.1551  0.8940	&	& 180.0 \\
	H   3 1.095  1 111.018  2 299.7	& H  0.4825 -1.1551  0.8940	&	& 299.7 \\
	H   3 1.095  1 110.996  2  60.2	& H  0.4723 -1.1665 -0.8706	&	& 180.0 \\
	H   3 1.095  1 110.265  2 180.0	& H -2.0542 -0.7710 -0.9003	&	& 60.2 \\
	H   4 1.095  2 111.009  1 299.7	& H -2.0651 -0.7856  0.8742	&	&	\\
	H   4 1.095  2 110.262  1 180.0	& H -2.7203  0.6060 -0.0058	&	&	\\
	H   4 1.095  2 110.989  1  60.2	& H  2.0542  0.7709 -0.9003	&	&	\\
									& H  2.7202 -0.6062 -0.0059	&	&	\\
									& H  2.0652  0.7854  0.8743	&	&	\\
	\bottomrule
	\end{tabular}
\caption{The physical system of \progname{}, using butane as an example 
molecule. The first three columns represent equivalent geometries, given in 
three different formats. The last column is the list of dihedral angles to be 
optimised by the program. During optimization, it is 
important that the 
dihedral angles map back to the input geometry using the same ordering for the 
atoms.}
\label{input-butane}
\end{table}

\subsubsection{Goal Statements}

\noindent Given an input geometry and a list of convergence criteria for a 
molecule, the goal statement is:

\begin{itemize}

\item[GS\refstepcounter{goalnum}\thegoalnum \label{goal}:] Return $n_G$ 
geometries that represent the maximum $Fit_G$ value that was found after the 
optimization. 

\item[GS\refstepcounter{goalnum}\thegoalnum \label{goal-energy}:] Return the 
energies of each geometry in the set that represents the maximum $Fit_G$ value 
from the optimization.

%To 
%create a set of conformational isomers that represent the local minima of the 
%potential energy surface by optimizing the molecule's 
%dihedral angles. 
%Alternatively, this GS can be described as: maximize the fitness function, 
%$Fit_G$.

\wss{I'm confused by the output being a set.  How many elements in the set?  All
  of them?  All of the ones the calculation can find?}

\end{itemize}

\subsection{Solution Characteristics Specification}

The instance models that govern \progname{} are presented in
Subsection~\ref{sec_instance}.  The information to understand the meaning of the
instance models and their derivation is also presented, so that the instance
models can be verified.

\subsubsection{Assumptions}

This section simplifies the original problem and helps in developing the
theoretical model by filling in the missing information for the physical
system. The numbers given in the square brackets refer to the theoretical model
[T], general definition [GD], data definition [DD], instance model [IM], or
likely change [LC], in which the respective assumption is used.

\begin{itemize}

%\item[A\refstepcounter{assumpnum}\theassumpnum \label{A_meaningfulLabel}:]

%\item[A\refstepcounter{assumpnum}\theassumpnum \label{A:one-min}:] Convergence 
%can be achieved when calculating the energy of the initial geometry, however 
%specified.

\item[A\refstepcounter{assumpnum}\theassumpnum \label{A:init-params-conv}:] The 
initial bond lengths and bond angles, regardless of the type of input 
specification, will allow for the energy calculations to converge. [\tref{T_SE}]

\item[A\refstepcounter{assumpnum}\theassumpnum \label{A:E-calculable}:] The 
energy of the molecule is calculable (using quantum mechanics). [\tref{T_SE}]

\item[A\refstepcounter{assumpnum}\theassumpnum \label{A:one-min}:] The 
potential energy surface can be represented by a real-valued, continuous 
function that contains at least one minimum. [\iref{IM:fitg}]

\item[A\refstepcounter{assumpnum}\theassumpnum \label{A:emp-func}:] The 
viability of a set of conformer geometries can be evaluated by solving for the 
fitness function, $Fit_G$, whose inputs depend on the RMSD distances between 
the conformers and the energies of those conformers. A bigger value for this 
function implies a better set of conformers. 
[\iref{IM:fitg}][\ddref{Sum_E}][\ddref{Sum_RMSD}]

\item[A\refstepcounter{assumpnum}\theassumpnum \label{A:linear-fit}:] The 
fitness function is a linear combination of the sum of energies and the sum of 
RMSD values. [\iref{IM:fitg}][\lcref{LC_linear-fit}]

\item[A\refstepcounter{assumpnum}\theassumpnum \label{A:conf=min}:] Conformers 
correspond to local minima on the potential energy surface. [\iref{IM:fitg}]

\meow{A\ref{A:conf=min} might be too similar to A\ref{A:stability}.}.

\item[A\refstepcounter{assumpnum}\theassumpnum \label{A:stability}:] Within one 
potential well, the molecular geometry corresponding to a lower (more negative) 
energy will be more stable than a geometry of a higher energy. [\iref{IM:fitg}]

\item[A\refstepcounter{assumpnum}\theassumpnum \label{A:dihedral-only}:] 
Conformers can be found by manipulating dihedral angles only (i.e. bond lengths 
and bond angles will be held fixed). [\iref{IM:fitg}]

\item[A\refstepcounter{assumpnum}\theassumpnum \label{A:fixed-mol}:] The 
molecular composition does not change during the 
optimization [\tref{T_SE}]

\item[A\refstepcounter{assumpnum}\theassumpnum \label{A:one-mol}:] The 
molecule is the only calculable object in the system (no solvent, other 
molecules, etc.). This assumption does not exclude guest-host chemistry, or 
non-bonded ``molecules", from the list of permissible inputs. [\iref{IM:fitg}]

\item[A\refstepcounter{assumpnum}\theassumpnum \label{A:atom-ordering}:] The 
conformer space is independent of the ordering of the input atoms. This may 
change (subject to proof/benchmarking). 
[\iref{IM:fitg}][\lcref{LC_indep-ordering}]

%\item[A\refstepcounter{assumpnum}\theassumpnum \label{A:one-min}:]

\end{itemize}

\subsubsection{Theoretical Models}\label{sec_theoretical}

This section focuses on the general equations and laws that \progname{} is based
on. A quantum chemistry approach is used since this level of theory allows for 
a more accurate calculation of the molecular energy. Some assumptions are 
needed for these energy calculations in order to make the equations solvable, 
which will be discussed briefly here.

~\newline

\noindent
\begin{minipage}{\textwidth}
	\renewcommand*{\arraystretch}{1.5}
	\begin{tabular}{| p{\colAwidth} | p{\colBwidth}|}
		\hline
		\rowcolor[gray]{0.9}
		Number& T\refstepcounter{theorynum}\thetheorynum \label{T_SE}\\
		\hline
		Label&\bf Non-Relativistic Time-Independent Schr\"{o}dinger Equation \\
		\hline
		Equation&  $\hat{H}\ket{\Psi} = E\ket{\Psi}$ \\
		\hline
		Description & 
		Most quantum chemical methods are focused on solving the above equation 
		for a molecule with some number of nuclei and electrons. This equation 
		is essentially an eigenvalue problem, where $E$ gives the energy of a 
		system (the eigenvalues), $\Psi$ is the wavefunction, and $\hat{H}$ is 
		the Hamiltonian operator. An example of the electronic Hamiltonian 
		operator can be found in the Appendix (Section \ref{appendix}).\\
		\hline
		Source & \cite{szabo-ostlund} \\
		\hline
		Ref.\ By & \ddref{Sum_E}\\
		\hline
	\end{tabular}
\end{minipage}\\

The Hamiltonian, in its most explicit form, is a sum of the kinetic energies of 
the electrons ($\hat{T_e}$), the kinetic energies of the nuclei ($\hat{T_n}$), 
the electron-electron potential energy ($\hat{V_{ee}}$), the electron-nuclear 
potential energy ($\hat{V_{en}}$), and the nuclear-nuclear potential energy 
($\hat{V_{nn}}$). The potential energy terms arise from Couloumbic repulsive 
and attractive forces. The way in which the Schr\"{o}dinger equation is solved 
and how the Hamiltonian is constructed are described by the quantum chemical 
method (QCM) that is used. Some QCM examples include Hartree-Fock, 
Coupled-Cluster, Configuration-Interaction, perturbation theory, and density 
functional theory.

The Born-Oppenheimer approximation is commonly used in quantum chemical 
methods. The nuclear spatial bounds, as set by its wavefunction, are on
the order of 100,000 times larger than those of the electron. Due to this
size difference, the nuclei move much 
more slowly. As a result, the nuclei are considered fixed and only the 
electrons move in the system. Therefore, the $\hat{T_n}$ term of the 
Hamiltonian is neglected and the $\hat{V_{nn}}$ term becomes constant. When a 
constant is added to an operator, the eigenfunctions do not change 
(wavefunction is the same). The eigenvalues (the energies) are added to the 
constant to get the final result. The electronic Hamiltonian in the 
Appendix is shown explicitly after these assumptions have been applied.

The wavefunction has no physical meaning; it is a complex-valued function for a 
single particle where the input is the position vector. The wavefunction 
squared ($|\psi(r)^2|$) is proportional to the probability of finding the 
particle at position $r = X,Y,Z$. That is, the integration over all space for 
$|\psi(r)^2|$ is equal to one, because the particle (if it exists) must be 
somewhere in space.

For an electron, the wavefunction describes a hydrogen atomic orbital (AO), and 
its inputs are the spin ($\sigma$) of the electron ($\alpha$ or $\beta$) and 
the position ($r$). For a multi-electron wavefunction, we have a linear 
combination of atomic orbitals (LCAO) to give the molecular orbital (MO). Since 
the number of orbitals needed to exactly describe a molecule is infinite, the 
MO is approximated by contracting the number of AO to a finite set (and thus a 
finite space), affording $\Psi(r, \sigma) = \sum\limits_{i}c_i 
\phi(r,\sigma)_i$, where $\phi(r,\sigma)_i$ is the ith AO and $c_i$ is a 
constant  
described by the basis set.

There are two main types of AO in quantum chemistry - Slater-type orbitals 
(STO) and Gaussian-type orbitals (GTO). STO have the form: $\phi(r,\sigma) = 
c_Ne^{-\zeta r}$, whereas GTO have the form: $\phi(r,\sigma) = c_Ne^{-\zeta 
r^2}$, 
where $c_N$ is a value that depends on normalization and 
the principle quantum 
number, $\zeta$ is a constant related to the effective nuclear charge, and r is 
the distance of the electron from the nucleus. STO are more accurate, have poor 
near-nuclear behaviour (approach $\infty$), but are expensive in terms of the 
evaluation of the integrals. GTO, on the other hand, are much faster to 
evaluate, but are less accurate. When a basis set is chosen for a quantum 
chemical calculation, the set of one-particle functions (AO) used to build the 
MO are described. Most quantum chemistry packages use GTO, and often the 
solution for a more accurate MO is to use a linear combination of GTO to mimic 
one STO.

~\newline

\subsubsection{General Definitions}\label{sec_gendef}

This section collects the laws and equations that will be used in deriving the
data definitions, which in turn are used to build the instance models.

~\newline

\noindent
\begin{minipage}{\textwidth}
\renewcommand*{\arraystretch}{1.5}
\begin{tabular}{| p{\colAwidth} | p{\colBwidth}|}
\hline
\rowcolor[gray]{0.9}
Number& GD\refstepcounter{defnum}\thedefnum \label{GD_RMSD}\\
\hline
Label &\bf Root-mean square deviation \\
\hline
% Units&$MLt^{-3}T^0$\\
% \hline
Units&\si{\angstrom}\\
\hline
Equation&$RMSD_{ij} = \sqrt{\frac{1}{n_a}\sum\limits_{k=1}^{n_a} ((x_{ki} - 
x_{kj})^2+(y_{ki} - y_{kj})^2+(z_{ki} - z_{kj})^2)}$  \\
\hline
Description &
The root-mean square deviation ($RMSD$) is an average distance between the 
atoms of conformer $i$ and the atoms of conformer $j$. The inputs to this 
equation are the xyz coordinates for each atom (from both conformers) and the 
number of total atoms.
\\
& $n_a$ number of atoms in each conformer \\
& $x_{ki}$ the x-coordinate of the $k^{th}$ atom from conformer $i$ 
(\si{\angstrom}) 
\\
& $y_{kj}$ the y-coordinate of the $k^{th}$ atom from conformer $j$ 
(\si{\angstrom})
\\
& etc. \\
\hline
  Source & \\
  \hline
  Ref.\ By & \ddref{Sum_RMSD} \\
  \hline
\end{tabular}
\end{minipage}\\

\wss{Can GD1 be viewed as a refinement of T1?  If so, this should be pointed
  out.  If not, GD1 might make more sense as a data definition, or maybe as a
  theoretical model?}

\subsubsection{Data Definitions}\label{sec_datadef}

This section collects and defines all the data needed to build the instance
models. The dimension of each quantity is also given.

~\newline

\noindent
\begin{minipage}{\textwidth}
	\renewcommand*{\arraystretch}{1.5}
	\begin{tabular}{| p{\colAwidth} | p{\colBwidth}|}
		\hline
		\rowcolor[gray]{0.9}
		Number & DD\refstepcounter{datadefnum}\thedatadefnum \label{Sum_E}\\
		\hline
		Label & \bf Sum of conformer energies\\
		\hline
		Symbol & $S_E$\\
		\hline
		% Units& $Mt^{-3}$\\
		% \hline
		Units & \si{\joule}\\
		\hline
		Equation & $S_E = \left|\sum\limits_{i=1}^{n_G}E_i\right|$ \\
		\hline
		Description & 
		$E_i$ is the energy (\si{\joule}) of conformer $i$, as calculated by 
		solving the Schr\"{o}dinger equation (T\ref{T_SE}), and $n_G$ is the 
		number of conformers being simultaneously optimised by the system. The 
		individual energies should be negative; the absolute value bars imply 
		that, when this summation is included in the overall fitness function, 
		we are maximizing the value of the fitness 
		function.
		\\
		\hline
		Sources& N/A \\
		\hline
		Ref.\ By & \iref{IM:fitg}\\
		\hline
	\end{tabular}
\end{minipage}\\

\noindent
\begin{minipage}{\textwidth}
	\renewcommand*{\arraystretch}{1.5}
	\begin{tabular}{| p{\colAwidth} | p{\colBwidth}|}
		\hline
		\rowcolor[gray]{0.9}
		Number & DD\refstepcounter{datadefnum}\thedatadefnum \label{Sum_RMSD}\\
		\hline
		Label & \bf Sum of root-mean square deviations \\
		\hline
		Symbol & $S_\text{RMSD}$\\
		\hline
		% Units& $Mt^{-3}$\\
		% \hline
		Units & \si{\angstrom}\\
		\hline
		Equation & $S_\text{RMSD} = \sum\limits_{i\neq j}RMSD_{ij}$ \\
		\hline
		Description & 
		$RMSD_{ij}$ is the root-mean square deviation between conformers $i$ 
		and $j$ (\si{\angstrom}), as calculated by using the distance formula 
		given in \dref{GD_RMSD}. This 
		distance represents how different two conformer geometries are from one 
		another on average.
		\\
		\hline
		Sources& N/A \\
		\hline
		Ref.\ By & \iref{IM:fitg}\\
		\hline
	\end{tabular}
\end{minipage}\\

\subsubsection{Instance Models} \label{sec_instance}    

This section transforms the problem defined in Section~\ref{Sec_pd} into 
one which is expressed in mathematical terms. It uses concrete symbols defined 
in Section~\ref{sec_datadef} to replace the abstract symbols in the models 
identified in Sections~\ref{sec_theoretical} and~\ref{sec_gendef}.

The goal \gsref{goal} is solved by \iref{IM:fitg} 
. \iref{IM:fitg} is an empirical function designed to explore the 
  potential energy space for a molecule. Given a multi-variate surface that 
  depends on dihedral angles, \progname{} will manipulate those dihedral angles 
  and solve for 
  $Fit_G$. This fitness function has two arbitrary coefficients, $C_E$ and 
  $C_\text{RMSD}$ that the user must determine through experimentation with their 
  molecule.

~\newline

%Instance Model 1

\noindent
\begin{minipage}{\textwidth}
\renewcommand*{\arraystretch}{1.5}
\begin{tabular}{| p{\colAwidth} | p{\colBwidth}|}
  \hline
  \rowcolor[gray]{0.9}
  Number& IM\refstepcounter{instnum}\theinstnum \label{IM:fitg}\\
  \hline
  Label& \bf Fitness of conformer geometries $Fit_G$\\
  \hline
  Input&$C_E$, $C_\text{RMSD}$, $S_E$, $S_\text{RMSD}$ \\
  & The input is constrained so that $C_E > 0$ and $C_\text{RMSD} > 0$\\
  \hline
  Output&$Fit_G = C_E S_E + C_\text{RMSD} S_\text{RMSD}$ \\
  & This equation is entirely empirical and its output does not have any 
  physical meaning. The purpose of the equation is to represent the optimal set 
  of conformers for a given input molecule.\\
  \hline
  Description&$C_E$ is the energy coefficient 
  (\si{1/\joule}).\\
  &$S_E$ is the absolute value of the sum of conformer energies (\ddref{Sum_E}) 
  (\si{\joule}).\\
  &$C_\text{RMSD}$ is root-mean square deviation coefficient (\si{1/\angstrom}).\\
  &$S_\text{RMSD}$ is sum of root-mean square deviations for each conformer 
  (\ddref{Sum_RMSD}) (\si{\angstrom}).\\
  & The above equation applies when the number of conformers is greater than 
  one. 
  If the number of conformers is exactly one, then the $RMSD$ terms should be 
  set to zero. \wss{You mentioned multiple assumptions in the assumptions
    section that are relevant to this IM, but they do not actually appear in
    your description anywhere.  All assumptions that are listed should be
    ``invoked'' somewhere in your documentation.}
  \\
  \hline
  Sources& N/A \\
  \hline
  Ref.\ By & None \\
  \hline
\end{tabular}
\end{minipage}\\

\wss{I may have missed it, but I don't see an optimization anywhere.  I thought
  I would see an objective function that you are minimizing?  Something that
  will trace to your goal statement.  I also thought that the output of the IM
  would be a set of conformers.}

\subsubsection{Data Constraints} \label{sec_DataConstraints}    

Tables~\ref{TblInputVar} and \ref{TblOutputVar} show the data constraints on the
input and output variables, respectively.  The column for physical constraints gives
the physical limitations on the range of values that can be taken by the
variable.  The column for software constraints restricts the range of inputs to
reasonable values.  The constraints are conservative, to give the user of the
model the flexibility to experiment with unusual situations.  The column of
typical values is intended to provide a feel for a common scenario, but for 
this project the typical values are very specific to the input geometry.  The
uncertainty column provides an estimate of the confidence with which the
physical quantities can be measured.  This information would be part of the
input if one were performing an uncertainty quantification exercise.

The specification parameters in Table~\ref{TblInputVar} are listed in
Table~\ref{TblSpecParams}.

\meow{There is no associated uncertainty with any of my values. Also, in most 
cases the typical values will be specific to the molecule (specifically, number 
of atoms). Should these two columns still be in this table?}

For $C_E$ and $C_\text{RMSD}$, the values for the coefficients depend on the shape 
of the potential energy surface. For example, in a surface where the minima are 
close together and the potential wells for the conformers are very high, 
placing more emphasis on $C_E$ (i.e. making $C_E$ bigger) would enable a better 
$Fit_G$. In the case where there are many low-lying conformers with wide energy 
basins, then the emphasis on $C_\text{RMSD}$ would afford a better result. The 
number of conformers may not be known at the start of the program; the user may 
have to determine this value through experimentation with the code.

\begin{table}[!h]
  \caption{Input Variables} \label{TblInputVar}
  \renewcommand{\arraystretch}{1.2}
\noindent \begin{longtable*}{l l l l c} 
  \toprule
  \textbf{Var} & \textbf{Physical Constraints} & \textbf{Software Constraints} &
                             \textbf{Typical Value} & \textbf{Uncertainty}\\
  \midrule 
  $n_G$ & $n\geq 2\text{*} | n\in{\mathbb{Z}}$ & $n_G \leq n_{max}$ & 2-5 & N/A 
  \\
  $C_E$ & $C_E > 0$ & $C_E > 0$ & 0.5 & N/A \\
  $C_\text{RMSD}$ & $C_\text{RMSD} > 0$ & $C_\text{RMSD} > 0$ & 0.5 & N/A \\
  $BS$ & available for molecule & available in software package & cc-pVTZ & N/A 
  \\
  $QCM$& available for molecule & available in software package & CCSD & N/A \\
  \bottomrule
\end{longtable*}
\end{table}

\noindent 
\begin{description}
\item[(*)] if $n_G$ is equal to one, then the $Fit_G$ function should be 
changed to only consider the energy term and not the RMSD term.
\end{description}

\begin{table}[!h]
\caption{Specification Parameter Values} \label{TblSpecParams}
\renewcommand{\arraystretch}{1.2}
\noindent \begin{longtable*}{l l} 
  \toprule
  \textbf{Var} & \textbf{Value} \\
  \midrule 
  None & - \\
  \bottomrule
\end{longtable*}
\end{table}

\begin{table}[!h]
\caption{Output Variables} \label{TblOutputVar}
\renewcommand{\arraystretch}{1.2}
\noindent \begin{longtable*}{l l} 
  \toprule
  \textbf{Var} & \textbf{Physical Constraints} \\
  \midrule 
  $Fit_G$ & $Fit_G > 0$
  \\
  \bottomrule
\end{longtable*}
\end{table}

\subsubsection{Properties of a Correct Solution} \label{sec_CorrectSolution}

\noindent
This problem will return its ``best guess'' as to the most favoured 
conformations for the input molecule. The correct solution should return unique 
conformers, rather than $n_G$ copies of the global minimum energy conformer. 
During the optimization, many potential solutions will be generated. A solution 
consists of $n_G$ geometries, where each geometry $G_i$ is generated from the 
dihedral angles list $D_i$ and the initial geometry specification (which 
dictates bond angles and bond lengths). The energy should be calculated for 
each of these geometries. The best of these potential solutions exhibits the 
property that it has the biggest $Fit_G$ value of all the sets of geometries 
currently available. The returned geometries should also converge when their 
energy calculations are performed.

\section{Requirements} \label{section-require}

This section provides the functional requirements, the business tasks that the
software is expected to complete, and the nonfunctional requirements, the
qualities that the software is expected to exhibit.

\subsection{Functional Requirements}




\noindent \begin{itemize}

\item[R\refstepcounter{reqnum}\thereqnum \label{R_Inputs}:] Input the following 
parameters for the molecule whose conformers should be found. See Table 
\ref{input-butane} for an example of what these molecular specifications look 
like.

\begin{tabularx}{\textwidth}{p{1.7cm}p{2cm}p{2cm}X}
	\toprule
	symbol & unit & data type & description \\
	\midrule
	$n_G$ & unitless & integer & number of conformers (distinct geometries) 
	to search for \\
	$C_E$ & \si{\joule} & floating point & coefficient for energy term in 
	fitness function \\
	$C_\text{RMSD}$ & \si{\metre} & floating point & coefficient for RMSD term 
	in fitness function \\
	BS & unitless & string & basis set \\
	QCM & unitless & string & quantum chemical method \\
	molecular geometry & \si{\angstrom}, \textdegree & file, string & name, 
	SMILES 
	string, xyz file, z-matrix file - a way to specify the connectivity of 
	the input molecule \\
	\bottomrule
\end{tabularx}

\item[R\refstepcounter{reqnum}\thereqnum \label{R_OutputInputs}:] Given the 
inputs from \rref{R_Inputs}, use \iref{IM:fitg} to solve for $Fit_G$. Based on 
the initial geometry, generate sets of random dihedral angles $D_i$, each with 
length of $n_a - 3$. The number of sets to generate should be equal to $n_G$.

\item[R\refstepcounter{reqnum}\thereqnum \label{R_Calculate}:] Calculate the 
energy $E_i$ for each conformer and the RMSD distance between each conformer 
pair and use these values to solve for \iref{IM:fitg}.

\item[R\refstepcounter{reqnum}\thereqnum \label{R_VerifyOutput}:] Verify that 
the energy calculations converge (\aref{A:init-params-conv}, 
\aref{A:E-calculable}) and that $Fit_G$ satisfies 
the requirement in 
Table \ref{TblOutputVar}.

\item[R\refstepcounter{reqnum}\thereqnum \label{R_Output}:] Generate output 
geometries by combining the dihedral angles $D_i$ for each conformer 
with the original geometric specifications (i.e. bond lengths, bond angles, and 
satisfying \aref{A:dihedral-only}, \aref{A:atom-ordering}).

\item[R\refstepcounter{reqnum}\thereqnum \label{R_template}:] Program contains 
a set of template files that the user can modify to run their own calculations.

\item[R\refstepcounter{reqnum}\thereqnum \label{R_opt}:] The geometries for the 
conformers are optimized such that the returned value of $Fit_G$ is maximized. 

\end{itemize}

\subsection{Nonfunctional Requirements}

\begin{enumerate}
	\item Given that the fitness function is 
	empirical, \progname{} should be relatively 
robust with regards to changes made to the definition 
of $Fit_G$.

\item Maintainability is also important for other 
students who will use the project 
later-on. 

\item The program should be parallelisable and capable 
of running on 
high-performance computing servers without a difficult install process 
(usability, portability). 

\item The program should be easy to use and quick to explain to chemists.

\item The program should work well with other quantum 
chemistry packages.

\end{enumerate}

\section{Likely Changes} \label{section-lc}

\noindent \begin{itemize}

\item[LC\refstepcounter{lcnum}\thelcnum\label{LC_linear-fit}:] The assumption 
that the fitness function is linear with respect to energies and distances has 
not been verified; the function could be exponential, sinusoidal, etc.
[\aref{A:linear-fit}]

\item[LC\refstepcounter{lcnum}\thelcnum\label{LC_indep-ordering}:] The 
assumption that the ordering of the atoms does not change the conformer space 
has not been verified. [\aref{A:atom-ordering}]

\end{itemize}

\section{Traceability Matrices and Graphs} \label{section-trace}

The purpose of the traceability matrices is to provide easy references on what
has to be additionally modified if a certain component is changed.  Every time a
component is changed, the items in the column of that component that are marked
with an ``X'' may have to be modified as well. 


% Table~\ref{Table:trace} shows the
%dependencies of theoretical models, general definitions, data definitions, and
%instance models with each other. Table~\ref{Table:R_trace} shows the
%dependencies of instance models, requirements, and data constraints on each
%other. Table~\ref{Table:A_trace} shows the dependencies of theoretical models,
%general definitions, data definitions, instance models, and likely changes on
%the assumptions.

%\afterpage{
%\begin{landscape}
\begin{table}[h!]
\centering
\label{Table:A_trace}
\begin{tabular}{|c|c|c|c|c|c|c|c|c|c|c|c|}
\hline
	& \aref{A:init-params-conv} 
	& \aref{A:E-calculable}
	& \aref{A:one-min}
	& \aref{A:emp-func}
	& \aref{A:linear-fit}
	& \aref{A:conf=min}
	& \aref{A:stability}
	& \aref{A:dihedral-only}
	& \aref{A:fixed-mol}
	& \aref{A:one-mol}
	& \aref{A:atom-ordering} \\
\hline
\tref{T_SE}               &X&X& & & & & & &X& & \\ \hline
\ddref{Sum_E}             & & & &X& & & & & & & \\ \hline
\ddref{Sum_RMSD}          & & & &X& & & & & & & \\ \hline
\dref{GD_RMSD}            & & & & & & & & & & & \\ \hline
\iref{IM:fitg}            & & &X&X&X&X&X&X& &X&X\\ \hline
\lcref{LC_linear-fit}     & & & & &X& & & & & & \\ \hline
\lcref{LC_indep-ordering} & & & & & & & & & & &X\\ \hline
\end{tabular}
\caption{Traceability Matrix Showing the Connections Between Assumptions and Other Items}

\end{table}
%\end{landscape}
%}



\begin{table}[H]
	\centering
	\label{Table:trace}
	\begin{tabular}{|c|c|c|c|c|c|}
		\hline        
		& \tref{T_SE} 
		& \ddref{Sum_E}
		& \ddref{Sum_RMSD}
		& \dref{GD_RMSD}
		& \iref{IM:fitg} \\
		\hline
		\tref{T_SE} & & & & & \\ \hline
		\ddref{Sum_E} & X & & & &  \\ \hline
		\ddref{Sum_RMSD} & & & & X & \\ \hline
		\dref{GD_RMSD} & & & & & \\ \hline
		\iref{IM:fitg} & & X & X & & \\
		\hline
	\end{tabular}
	\caption{Traceability Matrix Showing the Connections Between Items of 
	Different Sections}
\end{table}


\begin{table}[h!]
	\centering
	\begin{tabular}{|c|c|c|c|c|}
		\hline
		& \iref{IM:fitg}
		& \rref{R_Inputs} 
		& Table \ref{TblOutputVar}
		& \tref{T_SE} \\
		\hline
		\iref{IM:fitg}           & & & & \\ \hline
		\rref{R_Inputs}           & & & &\\ \hline
		\rref{R_OutputInputs}      & X & X & & \\ \hline
		\rref{R_Calculate}        & X & & & \\ \hline
		\rref{R_VerifyOutput}    & & & X & X\\ \hline
		\rref{R_Output}   & X & & & \\
		\hline
	\end{tabular}
	\caption{Traceability Matrix Showing the Connections Between Requirements 
	and Instance Models}
	\label{Table:R_trace}
\end{table}



%The purpose of the traceability graphs is also to provide easy references on
%what has to be additionally modified if a certain component is changed.  The
%arrows in the graphs represent dependencies. The component at the tail of an
%arrow is depended on by the component at the head of that arrow. Therefore, if 
%a
%component is changed, the components that it points to should also be
%changed. Figure~\ref{Fig_ATrace} shows the dependencies of theoretical models,
%general definitions, data definitions, instance models, likely changes, and
%assumptions on each other. Figure~\ref{Fig_RTrace} shows the dependencies of
%instance models, requirements, and data constraints on each other.

% \begin{figure}[h!]
% 	\begin{center}
% 		%\rotatebox{-90}
% 		{
% 			\includegraphics[width=\textwidth]{ATrace.png}
% 		}
% 		\caption{\label{Fig_ATrace} Traceability Matrix Showing the Connections Between Items of Different Sections}
% 	\end{center}
% \end{figure}


% \begin{figure}[h!]
% 	\begin{center}
% 		%\rotatebox{-90}
% 		{
% 			\includegraphics[width=0.7\textwidth]{RTrace.png}
% 		}
% 		\caption{\label{Fig_RTrace} Traceability Matrix Showing the Connections Between Requirements, Instance Models, and Data Constraints}
% 	\end{center}
% \end{figure}

\newpage

\bibliographystyle {plainnat}
%\bibliographystyle{ieeetr}
\bibliography {../../ReferenceMaterial/References}

\newpage

\section{Appendix} \label{appendix}

\noindent
\begin{minipage}{\textwidth}
	\renewcommand*{\arraystretch}{1.5}
	\begin{tabular}{| p{\colAwidth} | p{\colBwidth}|}
		\hline
		\rowcolor[gray]{0.9}
		%Number& T\refstepcounter{theorynum}\thetheorynum \label{T_HAMEE}\\
		Number & T\refstepcounter{theorynum}\thetheorynum \label{T_HAMEE}\\	
		\hline
		Label&\bf Electronic Hamiltonian \\
		\hline
		Equation&  $\hat{H_{ee}} = 
		\sum\limits_{i=1}^{N}\frac{-\hbar^2}{2m_e}\nabla_i^2 + 
		\sum\limits_{i=1}^{N} 
		\sum\limits_{\alpha=1}^{P}\frac{-Z_a 
			q_e^2}{4\pi\epsilon_0|\overrightarrow{r_i}-\overrightarrow{R_\alpha}|}
			 + 
		\sum\limits_{i=1}^{N}\sum\limits_{j=i+1}^{N}\frac{q_e^2}{4\pi\epsilon_0|\overrightarrow{r_i}-\overrightarrow{r_j}|}
		$ \\
		\hline
		Description & 
		The above equation gives the electronic 
		Hamiltonian operator for a 
		molecule with N-\ce{e-} and P-nuclei. $\hbar = \frac{h}{2\pi}$ is the 
		reduced form of Planck's constant, $m_e$ is the mass of an electron, 
		$\nabla$ is the gradient, $Z_a$ is the atomic number for the given 
		nuclei (a), $\epsilon_0$ is the permittivity of free space, $r_i$ are 
		the 
		xyz coordinates for the electrons, $R_\alpha$ are the xyz coordinates 
		for the nuclei, and $q_e$ is the charge of an electron.
		\\
		\hline
		Ref.\ By & \tref{T_SE}\\
		\hline
	\end{tabular}
\end{minipage}\\

\subsection{Symbolic Parameters}\label{num-const}

These symbols appear in the documentation and represent real numbers.

\renewcommand{\arraystretch}{1.2}
%\noindent \begin{tabularx}{1.0\textwidth}{l l X}
\noindent \begin{longtable*}{l l p{12cm}} \toprule
	\textbf{Symbol} & \textbf{Numerical Constant} & \textbf{Value}\\
	\midrule 
	$h$ & Planck's constant & 6.62607004x$10^{-34}$\si{\joule\second} \\
	$\hbar$ & reduced Planck's constant & $h/2\pi$\si{\joule\second} \\
	$m_e$ & mass of an electron & 9.10938x$10^{-31}$\si{\kilogram} \\
	$q_e$ & charge of an electron & 1.60217662x$10^{-19}$\si{\coulomb} \\
	$\epsilon_0$ & permittivity of free space  & 8.854187817x$10^{-12} 
	\si{\farad/\metre}$ \\
	
	\bottomrule
\end{longtable*}

\end{document}
